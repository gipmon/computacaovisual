\documentclass[pdftex,12pt,a4paper]{report}

\usepackage[portuguese,english]{babel}
\usepackage[T1]{fontenc} 
\usepackage[utf8]{inputenc}
\usepackage[pdftex]{graphicx}
\usepackage{minitoc}
\usepackage{hyperref}
\usepackage{indentfirst}
\usepackage[compact]{titlesec}
\usepackage{fancyhdr}
\usepackage{caption}
\usepackage{pgfplots}
\usepackage{pgfplotstable}
\usepackage{fixltx2e}
\usepackage{mathtools}
\usepackage{fancyhdr}
\usepackage{listings}
\usepackage{color}
\usepackage{sverb}
\usepackage[section]{placeins}

%Highlight
\newcommand{\shellcmd}[1]{\\\indent\indent\texttt{\footnotesize\# #1}\\}

\pagestyle{fancy}
\renewcommand*\thesection{\thechapter\arabic{section}}
\newcommand{\HRule}{\rule{\linewidth}{0.5mm}}
\begin{document}

\input{./titulo.tex} %Titulo do Relatorio
\renewcommand{\headrulewidth}{0pt}

%Cabeçalhos de rodapé
\fancyhead{}
\fancyfoot{}
\lhead{Puzzle}
\rhead{CV - 2015/2016}
\lfoot{\\ Rafael Ferreira nmec: 67405 \\ Rodrigo Cunha nmec: 67800}
\rfoot{\thepage}

%Renomear Comandos
\renewcommand*\contentsname{Conteúdos}
\renewcommand*\figurename{Figura}
\renewcommand*\tablename{Tabela}

%Conteúdos, dar paragrafo
\tableofcontents
%Headers
\renewcommand{\headrulewidth}{0.15pt}
\renewcommand{\thechapter}{}


\section{Introdução}
% o que, porquê e o objetivo
O trabalho proposto para o projeto da unidade curricular de Segurança é um IEDCS: Identity Enabled Distribution Control System. Para o  efeito foi necessário implementar uma Ebook Webstore, um WebService e um Player de reprodução dos Ebooks em formato de texto.

O objetivo deste sistema é garantir a máxima e possível segurança do serviço, utilizando os conhecimentos adquiridos na unidade curricular de Segurança. Para isso são necessários vários processos como por exemplo, a utilização de certificados HTTPS, a cifragem de todo o material existente, derivação de chaves e registo de utilizadores.	

O relatório reflete todos os passos e decisões tomadas na criação do sistema, assim como uma análise ao que foi mostrado na primeira apresentação e decisões que se tomaram depois desta, tecnologias utilizadas, descrição dos vários processos existentes e conclusão.

\section{JavaScript e WebGL}

\subsection{Arquitectura da solução}

\begin{figure}[!htb]
\center
 \includegraphics[width=50mm,scale=1]{classes.pdf}
 \caption{Processo de decifra de um livro}
 \label{fig:decifra_livro}
\end{figure}

Na primeira apresentação foram concebidos dois processos que foram abandonados após a sua apresentação. O primeiro processo apresentado foi o “Book content cipher process”. Este pretendia fazer uso de cifras simétricas e assimétricas de forma a ser possível obter uma File Key que 

\end{document}